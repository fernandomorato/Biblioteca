\documentclass[12pt, a4paper, twoside]{article}
\usepackage[T1]{fontenc}
\usepackage[utf8]{inputenc}
\usepackage{amssymb,amsmath}
\usepackage{comment}
\usepackage{datetime}
\usepackage[pdfusetitle]{hyperref}
\usepackage[all]{xy}
\usepackage{graphicx}
\addtolength{\parskip}{.5\baselineskip}

%aqui comeca o que eu fiz de verdade, o resto veio e eu to com medo de tirar
\usepackage{listings} %biblioteca pro codigo
\usepackage{color}    %deixa o codigo colorido bonitinho
\usepackage[landscape, left=1cm, right=1cm, top=1cm, bottom=2cm]{geometry} %pra deixar a margem do jeito que o brasil gosta

\definecolor{gray}{rgb}{0.4, 0.4, 0.4} %cor pros comentarios
%\renewcommand{\footnotesize}{\small} %isso eh pra mudar o tamanho da fonte do codigo
\setlength{\columnseprule}{0.2pt} %barra separando as duas colunas
\setlength{\columnsep}{15pt} %distancia do texto ate a barra

\lstset{ %opcoes pro codigo
breaklines=true,
keywordstyle=\color{blue},
commentstyle=\color{gray},
basicstyle=\footnotesize,
breakatwhitespace=true,
language=C++,
%frame=single, % nao sei se gosto disso ou nao
numbers=none,
rulecolor=\color{black},
showstringspaces=false
stringstyle=\color{blue},
tabsize=4,
basicstyle=\ttfamily\footnotesize, % fonte
literate={~} {$\sim$}{1} % ~ bonitinho
}

\title{[UNICAMP] Moratonistas}
\author{Bernardo Archegas, Fernando Morato e Luiz Oda}


\begin{document}
\twocolumn
\date{} %tira a data
\maketitle


\renewcommand{\contentsname}{Índice} %troca o nome do indice para indice
\tableofcontents


%%%%%%%%%%%%%%%%%%%%
%
% Estruturas
%
%%%%%%%%%%%%%%%%%%%%

\section{Estruturas}

\subsection{Compressao de Coordenadas
}
\begin{lstlisting}
// Classe para comprimir coordenadas

template <typename T>
class CoordinateCompression {
public:
	CoordinateCompression(const std::vector<T> &_v) {
		v = _v;
		std::sort(v.begin(), v.end());
		v.resize(std::unique(v.begin(), v.end()) - v.begin());
	}

	int size() { return (int) v.size(); }

	int operator() (T x) {
		return (int) (std::lower_bound(v.begin(), v.end(), x) - v.begin());
	}
private:
	std::vector<T> v;
};
\end{lstlisting}

\subsection{Fenwick Tree
}
\begin{lstlisting}

template <typename T>
class FenwickTree {
public:
	void init(int _n) {
		n = _n;
		bit.assign(n + 1, 0);
	}

	void init(vector<T> &v) {
		n = (int) v.size();
		bit.assign(n + 1, 0);
		for (int i = 1; i <= n; i++) {
			bit[i] += v[i - 1];
			if (i + (i & -i) <= n)
				bit[i + (i & -i)] += bit[i];
		}
	}

	void update(int i, T x) {
		for (; i <= n; i += i & -i)
			bit[i] += x;
	}

	T query(int i) {
		T ans = 0;
		for (; i > 0; i -= i & -i)
			ans += bit[i];
		return ans;
	}

private:
	int n;
	std::vector<T> bit;
};
\end{lstlisting}

\subsection{Otimizacao para Mo - Hilbert
}
\begin{lstlisting}
// Source: https://codeforces.com/blog/entry/61203?#comment-522213
namespace MO {
	constexpr int logn = 20;
	constexpr int maxn = 1 << logn;

	long long hilbertorder(int x, int y) {
		long long d = 0;
		for (int s = 1 << (logn - 1); s; s >>= 1) {
			bool rx = x & s, ry = y & s;
			d = ((((d << 2) | rx) * 3) ^ static_cast<int>(ry));
			if (!ry) {
				if (rx) {
					x = maxn - x;
					y = maxn - y;
				}
				swap(x, y);
			}
		}
		return d;
	}

	template <class T>
	void sortQueries(vector<T> &v) {
		for (auto &x : v)
			x.id = hilbertorder(x.l, x.r);
		sort(v.begin(), v.end());
	}

	struct Query {
		// maybe add new stuff here
		long long id;
		int l, r, idx;

		// Query(int _idx, int _l, int _r): idx(_idx), l(_l), r(_r) {}

		bool operator < (const Query &o) const {
			return id < o.id;
		}
	};
}

using namespace MO;
\end{lstlisting}

\subsection{Policy Based Data Structure
}
\begin{lstlisting}
#include <ext/pb_ds/assoc_container.hpp>
#include <ext/pb_ds/tree_policy.hpp>

template<typename T, typename M = __gnu_pbds::null_type>
using ordered_set = __gnu_pbds::tree<T, M, less<T>, __gnu_pbds::rb_tree_tag, __gnu_pbds::tree_order_statistics_node_update>;

// Supports the same operations as the std::set

// find_by_order(k) - Returns an iterator to the k-th largest element (counting from 0)

// order_of_key(x) - Returns the number of items in a set that are strictly smaller than x
\end{lstlisting}

\subsection{Segment Tree
}
\begin{lstlisting}
struct Node {
	// atributes

	Node() {
		// empty node
	}

	Node() {
		// constructor
	}
 
	Node(const Node &l, const Node &r) {
		// merge
	}
};

template <class node_t, class e_t>
class SegTree {
public:
	void init(vector<e_t> &v) {
		n = (int) v.size();
		tree.resize(2 * n);
		for (int i = 0; i < n; i++) {
			tree[i + n] = node_t(v[i]);
		}
		for (int i = n - 1; i > 0; i--) {
			tree[i] = node_t(tree[2 * i], tree[2 * i + 1]);
		}
	}

	void update(int p, e_t x) {
		p += n;
		tree[p] = node_t(x);
		for (p /= 2; p > 0; p /= 2) {
			tree[p] = node_t(tree[2 * p], tree[2 * p + 1]);
		}
	}

	node_t query(int l, int r) {
		node_t ln, rn;
		for (l += n, r += n; l < r; l /= 2, r /= 2) {
			if (l & 1) ln = node_t(ln, tree[l++]);
			if (r & 1) rn = node_t(tree[--r], rn);
		}
		return node_t(ln, rn);
	}

private:
	int n;
	std::vector<node_t> tree;
};
\end{lstlisting}

\subsection{Segment Tree com Lazy
}
\begin{lstlisting}
struct Lazy {
	// atributes

	Lazy() {
		// constructor
	}

	void reset() {

	}

	void operator += (Lazy o) {
		// merge
	}
};

struct Node {
	// atributes

	Node() {
		// empty node
	}

	Node() {
		// constructor
	}
 
	Node(const Node &l, const Node &r) {
		// merge
	}

	void apply(Lazy lazy) {

	}
};

// may be changed to iterative
template <class node_t, class e_t, class lazy_t = int>
class SegTree {
public:
	void init(vector<e_t> &_v) {
		v = _v;
		n = (int) v.size();
		tree.resize(4 * n + 1);
		lazy.resize(4 * n + 1);
		dirty.assign(4 * n + 1, false);
		build(1, 0, n - 1);
	}

	void update(int node, int l, int r, int ql, int qr, lazy_t x) {
		push(node, l, r);
		if (ql > r || l > qr)
			return;
		if (ql <= l && r <= qr) {
			apply(node, l, r, x);
			push(node, l, r);
			return;
		}
		int m = (l + r) / 2;
		update(2 * node, l, m, ql, qr, x);
		update(2 * node + 1, m + 1, r, ql, qr, x);
		tree[node] = node_t(tree[2 * node], tree[2 * node + 1]);
	}

	node_t query(int node, int l, int r, int ql, int qr) {
		push(node, l, r);
		if (ql > r || l > qr)
			return node_t();
		if (ql <= l && r <= qr)
			return tree[node];
		int m = (l + r) / 2;
		return node_t(query(2 * node, l, m, ql, qr), query(2 * node + 1, m + 1, r, ql, qr));
	}
private:
	int n;
	std::vector<e_t> v;
	std::vector<bool> dirty;
	std::vector<node_t> tree;
	std::vector<lazy_t> lazy;

	void apply(int node, int l, int r, Lazy &lz) {
		tree[node].apply(lz);
		if (l != r) {
			dirty[node] = true;
			lazy[node] += lz;
		}
	}

	void push(int node, int l, int r) {
		if (dirty[node]) {
			int m = (l + r) / 2;
			apply(2 * node, l, m, lazy[node]);
			apply(2 * node + 1, m + 1, r, lazy[node]);
			lazy[node].reset();
			dirty[node] = false;
		}
	}

	void build(int node, int l, int r) {
		if (l == r)  {
			tree[node] = node_t(v[l]);
		} else {
			int m = (l + r) / 2;
			build(2 * node, l, m);
			build(2 * node + 1, m + 1, r);
			tree[node] = node_t(tree[2 * node], tree[2 * node + 1]);
		}
	}
};
\end{lstlisting}

\subsection{Segment Tree Persistente
}
\begin{lstlisting}

struct Node{
	int v = 0;
	Node *l = this, *r = this;
};

int CNT = 1;
Node buffer[MAXN * 20];
 
Node* update(Node *root, int l, int r, int idx, int val){
	Node *node = &buffer[CNT++];
	*node = *root;
	int mid = (l + r) / 2;
	node->v += val;
	if(l + 1 != r){
		if(idx < mid) node->l = update(root->l, l, mid, idx, val);
		else node->r = update(root->r, mid, r, idx, val);
	}
	return node;
}
 
int query(Node *node, int tl, int tr, int l, int r){
	if(l <= tl && tr <= r) return node->v;
	if(tr <= l || tl >= r) return 0;
	int mid = (tl + tr) / 2;
	return query(node->l, tl, mid, l, r) + query(node->r, mid, tr, l, r);
}
\end{lstlisting}

\subsection{Sparse Table
}
\begin{lstlisting}
// Based on Juvitus and tfg implementations
template <class T, class F = function<T(const T&, const T&)>>
class SparseTable {
public:
	void init(const vector<T> &v, const F &_f) {
		f = _f;
		n = (int) v.size();
		int l = 0;
		while ((1 << l) / 2 < n) {
			l++;
		}
		pw.assign(n + 1, -1);
		tab.assign(l, vector<T>(n));
		for (int i = 0; i < n; i++) {
			tab[0][i] = v[i];
			pw[i + 1] = pw[(i + 1) / 2] + 1;
		}
		for (int i = 0; i + 1 < l; i++) {
			for (int j = 0; j + (1 << i) < n; j++) {
				tab[i + 1][j] = f(tab[i][j], tab[i][j + (1 << i)]);
			}
		}
	}

	T query(int l, int r) {
		int k = pw[r - l];
		return f(tab[k][l], tab[k][r - (1 << k)]);
	}

private:
	F f;
	int n;
	vector<int> pw;
	vector<vector<T>> tab;
};
\end{lstlisting}

\subsection{Treap Implicita
}
\begin{lstlisting}
namespace treap {
	// mt19937 rng((unsigned int) chrono::steady_clock::now().time_since_epoch().count());

	struct node {
		node *l, *r;
		int value;
		int prior, sz;
		bool rev;

		node (int _v): value(_v) {
			sz = 1;
			prior = rng();
			rev = false;
			l = r = NULL;
		}
	};

	inline int getSize(node *nd) {
		return (nd ? nd->sz : 0);
	}

	inline void update(node *&nd) {
		nd->sz = getSize(nd->l) + getSize(nd->r) + 1;
	}

	void push(node *nd) {
		if (nd && nd->rev) {
			nd->rev = false;
			swap(nd->l, nd->r);
			if (nd->l) {
				nd->l->rev ^= true;
			}
			if (nd->r) {
				nd->r->rev ^= true;
			}
		}
	}

	void merge(node *&root, node *l, node *r) {
		push(l), push(r);
		if (!l || !r) {
			root = (l ? l : r);
		} else {
			if (l->prior > r->prior) {
				merge(l->r, l->r, r);
				root = l;
			} else {
				merge(r->l, l, r->l);
				root = r;
			}
			update(root);
		}
	}

	void split(node *root, node *&l, node *&r, int pos) {
		if (!root) {
			l = r = NULL;
		} else {
			push(root);
			int p = getSize(root->l) + 1;
			if (p <= pos) {
				split(root->r, root->r, r, pos - p);
				l = root;
			} else {
				split(root->l, l, root->l, pos);
				r = root;
			}
			update(root);
		}
	}

	void insert(node *&root, int value, int pos) {
		node *l = NULL, *r = NULL, *aux = new node(value);
		split(root, l, r, pos);
		merge(l, l, aux);
		merge(root, l, r);
	}

	void reverse(node *&root, int ll, int rr) {
		// reverses interval [ll, rr]
		node *l = NULL, *r = NULL, *k = NULL;
		split(root, l, r, ll - 1);
		split(r, k, r, rr - ll + 1);
		k->rev ^= true;
		merge(r, k, r);
		merge(root, l, r);
	}

	void shift(node *&root, int ll, int rr) {
		// right-cyclic-shift on interval [ll, rr]
		node *l = NULL, *r = NULL, *k = NULL, *t = NULL;
		split(root, l, r, ll - 1);
		split(r, k, r, rr - ll + 1);
		split(k, k, t, rr - ll);
		merge(k, t, k);
		merge(r, k, r);
		merge(root, l, r);
	}

	int query(node *&root, int pos) {
		// returns which node is at position "pos"
		node *l = NULL, *r = NULL, *k = NULL;
		split(root, l, r, pos);
		split(l, l, k, pos - 1);
		int ans = k->value;
		merge(l, l, k);
		merge(root, l, r);
		return ans;
	}
}
\end{lstlisting}

\subsection{Union Find
}
\begin{lstlisting}

class DSU {
public:
	void init(int n) {
		p.resize(n);
		std::iota(p.begin(), p.end(), 0);
		sz.assign(n, 1);
	}

	int getSize(int x) { return sz[x]; }

	int find(int x) { return x == p[x] ? x : p[x] = find(p[x]); }

	bool sameSet(int a, int b) { return find(a) == find(b); }

	void join(int a, int b) {
		a = find(a), b = find(b);
		if (a != b) {
			if (sz[a] > sz[b])
				std::swap(a, b);
			op.emplace_back(a, p[a]);
			os.emplace_back(b, sz[b]);
			p[a] = b;
			sz[b] += sz[a];
		}
	}

	void rollback() {
		assert(!op.empty() && !os.empty());
		p[op.back().first] = op.back().second;
		op.pop_back();
		sz[os.back().first] = os.back().second;
		os.pop_back();
	}

private:
	std::vector<int> p, sz;
	std::vector<std::pair<int, int>> op, os;
};
\end{lstlisting}



%%%%%%%%%%%%%%%%%%%%
%
% Geometria
%
%%%%%%%%%%%%%%%%%%%%

\section{Geometria}

\subsection{Convex Hull
}
\begin{lstlisting}
vector<PT> convexHull(vector<PT> &p, bool sorted = false) {
	int n = (int) p.size(), k = 0;
	if (n == 1)
		return p;
	if (!sorted)
		sort(p.begin(), p.end());
	vector<PT> h(2 * n + 1);
	// Upper-Hull
	for (int i = 0; i < n; i++) {
		while (k >= 2 && (p[i] - h[k - 2]) % (h[k - 1] - h[k - 2]) >= 0)
			k--;
		h[k++] = p[i];
	}
	// Lower-Hull
	for (int i = n - 2, t = k + 1; i >= 0; i--) {
		while (k >= t && (p[i] - h[k - 2]) % (h[k - 1] - h[k - 2]) >= 0)
			k--;
		h[k++] = p[i];
	}
	h.resize(k - 1);
	return h;
}
\end{lstlisting}

\subsection{Ploblema dos pares mais proximos
}
\begin{lstlisting}
pii closestPair(vector<PT> &p) {
	auto sqDist = [](PT pt) { return pt * pt; };
	long long dist = sqDist(p[0] - p[1]);
	pii ans(0, 1);
	int n = (int) p.size();
	vector<int> v(n);
	iota(v.begin(), v.end(), 0);
	sort(v.begin(), v.end(), [&](int a, int b) { return p[a].x < p[b].x; });
	set<pll> st;
	auto sq = [](long long x) { return x * x; };
	for (int l = 0, r = 0; r < n; r++) {
		while (sq(p[v[l]].x - p[v[r]].x) > dist) {
			st.erase(pll(p[v[l]].y, v[l]));
			l++;
		}
		long long delta = (ll) sqrt(dist) + 1;
		auto il = st.lower_bound(pll(p[v[r]].y - delta, -1));
		auto ir = st.lower_bound(pll(p[v[r]].y + delta, n + 1));
		for (auto it = il; it != ir; it++) {
			long long distNow = sqDist(p[v[r]] - p[it->second]);
			if (distNow < dist) {
				dist = distNow;
				ans = pii(v[r], it->second);
			}
		}
		st.insert(pll(p[v[r]].y, v[r]));
	}
	if (ans.first > ans.second)
		swap(ans.first, ans.second);
	return ans;
}
\end{lstlisting}

\subsection{Ponto 2D
}
\begin{lstlisting}
// template <typename T>
struct PT {
	#define T long long
	T x, y;
	PT(T _x = 0, T _y = 0): x(_x), y(_y) {}
	PT operator + (const PT &p)		const { return PT(x + p.x, y + p.y); }
	PT operator - (const PT &p) 	const { return PT(x - p.x, y - p.y); }
	PT operator * (T c)				const { return PT(c * x, c * y); }
	T operator * (const PT &p) 		const { return x * p.x + y * p.y; }
	T operator % (const PT &p) 		const { return x * p.y - y * p.x; }
	bool operator < (const PT &p) 	const { return x == p.x ? y < p.y : x < p.x; }
	bool operator == (const PT &p) 	const { return x == p.x && y == p.y; }

	friend std::ostream& operator << (std::ostream &os, const PT &p) {
		return os << p.x << ' ' << p.y;
	}
	friend std::istream& operator >> (std::istream &is, PT &p) {
		return is >> p.x >> p.y;
	}
};
\end{lstlisting}



%%%%%%%%%%%%%%%%%%%%
%
% Grafos
%
%%%%%%%%%%%%%%%%%%%%

\section{Grafos}

\subsection{Algoritmo Hungaro
}
\begin{lstlisting}
// Resolve o problema de assignment (matriz n x n)
// Colocar os valores da matriz em 'a' (pode < 0)
// assignment() retorna um par com o valor do
// assignment minimo, e a coluna escolhida por cada linha
//
// O(n^3)
// 64c53e

// copied from: https://github.com/brunomaletta/Biblioteca/blob/master/Codigo/Problemas/hungarian.cpp

template<typename T> struct hungarian {
	int n;
	vector<vector<T>> a;
	vector<T> u, v;
	vector<int> p, way;
	T inf;

	hungarian(int n_) : n(n_), u(n+1), v(n+1), p(n+1), way(n+1) {
		a = vector<vector<T>>(n, vector<T>(n));
		inf = numeric_limits<T>::max();
	}
	pair<T, vector<int>> assignment() {
		for (int i = 1; i <= n; i++) {
			p[0] = i;
			int j0 = 0;
			vector<T> minv(n+1, inf);
			vector<int> used(n+1, 0);
			do {
				used[j0] = true;
				int i0 = p[j0], j1 = -1;
				T delta = inf;
				for (int j = 1; j <= n; j++) if (!used[j]) {
					T cur = a[i0-1][j-1] - u[i0] - v[j];
					if (cur < minv[j]) minv[j] = cur, way[j] = j0;
					if (minv[j] < delta) delta = minv[j], j1 = j;
				}
				for (int j = 0; j <= n; j++)
					if (used[j]) u[p[j]] += delta, v[j] -= delta;
					else minv[j] -= delta;
				j0 = j1;
			} while (p[j0] != 0);
			do {
				int j1 = way[j0];
				p[j0] = p[j1];
				j0 = j1;
			} while (j0);
		}
		vector<int> ans(n);
		for (int j = 1; j <= n; j++) ans[p[j]-1] = j-1;
		return make_pair(-v[0], ans);
	}
};
\end{lstlisting}

\subsection{Binary Lifting
}
\begin{lstlisting}
namespace BinaryLifting {
	const int MAXN = 2e5 + 5;
	const int LOG = 20;

	int h[MAXN];
	int anc[LOG][MAXN];

	void dfs(vector<vector<int>> &adj, int on, int par, int lvl = 0) {
		h[on] = lvl;
		anc[0][on] = par;
		for (int to : adj[on]) {
			if (to != par) {
				dfs(adj, to, on, lvl + 1);
			}
		}
	}

	void init(vector<vector<int>> &adj, int start = 0) {
		dfs(adj, start, start);
		for (int i = 1; i < LOG; i++) {
			for (int j = 0; j < (int) adj.size(); j++) {
				anc[i][j] = anc[i - 1][anc[i - 1][j]];
			}
		}
	}

	void up(int &x, int d) {
		for (int i = 0; i < LOG; i++) {
			if (d & (1 << i))
				x = anc[i][x];
		}
	}

	int getLCA(int a, int b) {
		if (h[a] > h[b]) {
			swap(a, b);
		}
		up(b, h[b] - h[a]);
		if (a == b) {
			return a;
		}
		for (int i = LOG - 1; i >= 0; i--) {
			if (anc[i][a] != anc[i][b]) {
				a = anc[i][a];
				b = anc[i][b];
			}
		}
		return anc[0][a];
	}

	int dist(int a, int b) {
		return h[a] + h[b] - 2 * h[getLCA(a, b)];
	}
}
\end{lstlisting}

\subsection{Dinic - Max Flow
}
\begin{lstlisting}
// by tfg50
// found at: https://github.com/tfg50/Competitive-Programming/blob/master/Biblioteca/Graph/MaxFlow.cpp
template <class T = int>
class Dinic {
public:
	struct Edge {
		Edge(int a, T b){to = a;cap = b;}
		int to;
		T cap;
	};

	Dinic(int _n) : n(_n) {
		edges.resize(n);
	}

	T maxFlow(int src, int sink) {
		T ans = 0;
		while(bfs(src, sink)) {
			// maybe random shuffle edges against bad cases?
			T flow;
			pt = std::vector<int>(n, 0);
			while((flow = dfs(src, sink))) {
				ans += flow;
			}
		}
		return ans;
	}

	void addEdge(int from, int to, T cap, T other = 0) {
		edges[from].push_back(list.size());
		list.push_back(Edge(to, cap));
		edges[to].push_back(list.size());
		list.push_back(Edge(from, other));
	}

	bool inCut(int u) const { return h[u] < n; }
	int size() const { return n; }
private:
	int n;
	std::vector<std::vector<int> > edges;
	std::vector<Edge> list;
	std::vector<int> h, pt;

	T dfs(int on, int sink, T flow = 1e9) {
		if(flow == 0) {
			return 0;
		} if(on == sink) {
			return flow;
		}
		for(; pt[on] < (int) edges[on].size(); pt[on]++) {
			int cur = edges[on][pt[on]];
			if(h[on] + 1 != h[list[cur].to]) {
				continue;
			}
			T got = dfs(list[cur].to, sink, std::min(flow, list[cur].cap));
			if(got) {
				list[cur].cap -= got;
				list[cur ^ 1].cap += got;
				return got;
			}
		}
		return 0;
	}

	bool bfs(int src, int sink) {
		h = std::vector<int>(n, n);
		h[src] = 0;
		std::queue<int> q;
		q.push(src);
		while(!q.empty()) {
			int on = q.front();
			q.pop();
			for(auto a : edges[on]) {
				if(list[a].cap == 0) {
					continue;
				}
				int to = list[a].to;
				if(h[to] > h[on] + 1) {
					h[to] = h[on] + 1;
					q.push(to);
				}
			}
		}
		return h[sink] < n;
	}
};
\end{lstlisting}

\subsection{Heavy-Light Decomposition
}
\begin{lstlisting}
class HLD {
public:
	void init(int n) {
		p.resize(n);
		h.resize(n);
		in.resize(n);
		sz.resize(n);
		adj.resize(n);
		head.resize(n);
	}

	void addEdge(int a, int b) {
		adj[a].push_back(b);
		adj[b].push_back(a);
	}

	void build(int root = 0) {
		t = 0;
		head[root] = p[root] = root;
		buildChains(root, root);
		buildHld(root, root);
	}

private:
	vector<int> p, h, in, sz, head;
	vector<vector<int>> adj;
	int t;

	void buildChains(int on, int par) {
		sz[on] = 1;
		p[on] = par;
		for (auto &to : adj[on]) {
			if (to == par) {
				swap(to, adj[on].back());
				continue;
			}
			h[to] = h[on] + 1;
			buildChains(to, on);
			sz[on] += sz[to];
			if (sz[to] > sz[adj[on][0]]) {
				swap(to, adj[on][0]);
			}
		}
	}

	void buildHld(int on, int par) {
		in[on] = t++;
		for (auto to : adj[on]) {
			head[to] = (to == adj[on][0] ? head[on] : to);
			buildHld(to, on);
		}
	}
};
\end{lstlisting}

\subsection{Min Cost Max Flow
}
\begin{lstlisting}
// by tfg50
// found at: https://github.com/tfg50/Competitive-Programming/blob/master/Biblioteca/Graph/MCMF.cpp
template <class T = int>
class MCMF {
public:
	struct Edge {
		Edge(int a, T b, T c) : to(a), cap(b), cost(c) {}
		int to;
		T cap, cost;
	};

	MCMF(int size) {
		n = size;
		edges.resize(n);
		pot.assign(n, 0);
		dist.resize(n);
		visit.assign(n, false);
	}

	std::pair<T, T> mcmf(int src, int sink) {
		std::pair<T, T> ans(0, 0);
		if(!SPFA(src, sink)) return ans;
		fixPot();
		// can use dijkstra to speed up depending on the graph
		while(SPFA(src, sink)) {
			auto flow = augment(src, sink);
			ans.first += flow.first;
			ans.second += flow.first * flow.second;
			fixPot();
		}
		return ans;
	}

	void addEdge(int from, int to, T cap, T cost) {
		edges[from].push_back(list.size());
		list.push_back(Edge(to, cap, cost));
		edges[to].push_back(list.size());
		list.push_back(Edge(from, 0, -cost));
	}
private:
	int n;
	std::vector<std::vector<int>> edges;
	std::vector<Edge> list;
	std::vector<int> from;
	std::vector<T> dist, pot;
	std::vector<bool> visit;

	/*bool dij(int src, int sink) {
		T INF = std::numeric_limits<T>::max();
		dist.assign(n, INF);
		from.assign(n, -1);
		visit.assign(n, false);
		dist[src] = 0;
		for(int i = 0; i < n; i++) {
			int best = -1;
			for(int j = 0; j < n; j++) {
				if(visit[j]) continue;
				if(best == -1 || dist[best] > dist[j]) best = j;
			}
			if(dist[best] >= INF) break;
			visit[best] = true;
			for(auto e : edges[best]) {
				auto ed = list[e];
				if(ed.cap == 0) continue;
				T toDist = dist[best] + ed.cost + pot[best] - pot[ed.to];
				assert(toDist >= dist[best]);
				if(toDist < dist[ed.to]) {
					dist[ed.to] = toDist;
					from[ed.to] = e;
				}
			}
		}
		return dist[sink] < INF;
	}*/

	std::pair<T, T> augment(int src, int sink) {
		std::pair<T, T> flow = {list[from[sink]].cap, 0};
		for(int v = sink; v != src; v = list[from[v]^1].to) {
			flow.first = std::min(flow.first, list[from[v]].cap);
			flow.second += list[from[v]].cost;
		}
		for(int v = sink; v != src; v = list[from[v]^1].to) {
			list[from[v]].cap -= flow.first;
			list[from[v]^1].cap += flow.first;
		}
		return flow;
	}

	std::queue<int> q;
	bool SPFA(int src, int sink) {
		T INF = std::numeric_limits<T>::max();
		dist.assign(n, INF);
		from.assign(n, -1);
		q.push(src);
		dist[src] = 0;
		while(!q.empty()) {
			int on = q.front();
			q.pop();
			visit[on] = false;
			for(auto e : edges[on]) {
				auto ed = list[e];
				if(ed.cap == 0) continue;
				T toDist = dist[on] + ed.cost + pot[on] - pot[ed.to];
				if(toDist < dist[ed.to]) {
					dist[ed.to] = toDist;
					from[ed.to] = e;
					if(!visit[ed.to]) {
						visit[ed.to] = true;
						q.push(ed.to);
					}
				}
			}
		}
		return dist[sink] < INF;
	}

	void fixPot() {
		T INF = std::numeric_limits<T>::max();
		for(int i = 0; i < n; i++) {
			if(dist[i] < INF) pot[i] += dist[i];
		}
	}
};
\end{lstlisting}



%%%%%%%%%%%%%%%%%%%%
%
% Matematica
%
%%%%%%%%%%%%%%%%%%%%

\section{Matematica}

\subsection{Combinatoria
}
\begin{lstlisting}
// remember to change BOUNDS accordingly!!!

using T = Mint<>;

T fat[MAXN], inv[MAXN];

void setup() {
	fat[0] = inv[0] = 1;
	for (int i = 1; i < MAXN; i++) {
		fat[i] = fat[i - 1] * i;
	}
	inv[MAXN - 1] = fexp(fat[MAXN - 1], MOD - 2);
	for (int i = MAXN - 2; i >= 1; i--) {
		inv[i] = inv[i + 1] * (i + 1);
	}
}

// C(n, k) = n choose k, number of ways to choose a set of k elements from a set of n elements
T C(int n, int k) {
	if (n < k || k < 0)
		return T(0);
	return fat[n] * inv[k] * inv[n - k];
}

// C(n) = n-th Catalan number - number of valid parenthesis sequences of size 2 * n
T C(int n) {
	return C(2 * n, n) - C(2 * n, n + 1);
}
\end{lstlisting}

\subsection{Division Trick
}
\begin{lstlisting}
// by tfg
// found at: https://github.com/tfg50/Competitive-Programming/blob/master/Biblioteca/Math/DivisionTrick.cpp
// O(sqrt(n))
for(int l = 1, r; l <= n; l = r + 1) {
	r = n / (n / l);
	// n / i has the same value for l <= i <= r
}
\end{lstlisting}

\subsection{FFT
}
\begin{lstlisting}
// chamar com vector<cplx> para FFT, ou vector<mint> para NTT
//
// O(n log(n))
// 64e51b

// copied from: https://github.com/brunomaletta/Biblioteca/blob/master/Codigo/Matematica/convolution.cpp

template<typename T> void fft(vector<T> &a, bool f, int N, vector<int> &rev){
	for (int i = 0; i < N; i++)
		if (i < rev[i])
			swap(a[i], a[rev[i]]);
	int l, r, m;
	vector<T> roots(N);
	for (int n = 2; n <= N; n *= 2){
	    T::fill_rt(f, n, N, roots);

		for (int pos = 0; pos < N; pos += n){
			l = pos+0, r = pos+n/2, m = 0;
			while (m < n/2){
				auto t = roots[m]*a[r];
				a[r] = a[l] - t;
				a[l] = a[l] + t;
				l++; r++; m++;
			}
		}
	}
	if (f) {
		auto invN = T(1)/N;
		for(int i = 0; i < N; i++) a[i] = a[i]*invN;
	}
}

template<typename T> vector<T> convolution(vector<T> &a, vector<T> &b) {
	vector<T> l(a.begin(), a.end());
	vector<T> r(b.begin(), b.end());
	int ln = l.size(), rn = r.size();
	int N = ln+rn-1;
	int n = 1, log_n = 0;
	while (n <= N) { n <<= 1; log_n++; }
	vector<int> rev(n);
	for (int i = 0; i < n; ++i){
		rev[i] = 0;
		for (int j = 0; j < log_n; ++j)
			if (i & (1<<j))
				rev[i] |= 1 << (log_n-1-j);
	}
	assert(N <= n);
	l.resize(n);
	r.resize(n);
	fft(l, false, n, rev);
	fft(r, false, n, rev);
	for (int i = 0; i < n; i++)
		l[i] *= r[i];
	fft(l, true, n, rev);
	return l;
}
\end{lstlisting}

\subsection{Inteiro Modular
}
\begin{lstlisting}
template <class T>
T fexp(T b, long long e) {
	T ans = T(1);
	for (; e > 0; e /= 2) {
		if (e & 1ll)
			ans *= b;
		b *= b;
	}
	return ans;
}

// maybe store factorial inverses to reduce division cost (?)
template <int mod = MOD>
struct Mint {
	int val;

	Mint(int x = 0): val(x < 0 ? x + mod : x) {}

	void operator += (Mint<mod> o) { *this = *this + o; }
	void operator -= (Mint<mod> o) { *this = *this - o; }
	void operator *= (Mint<mod> o) { *this = *this * o; }
	void operator /= (Mint<mod> o) { *this = *this / o; }
	Mint<mod> operator + (Mint<mod> o) { return val + o.val >= mod ? val + o.val - mod : val + o.val; }
	Mint<mod> operator - (Mint<mod> o) { return val - o.val < 0 ? val - o.val + mod : val - o.val; }
	Mint<mod> operator * (Mint<mod> o) { return (int) (1ll * val * o.val % mod); }
	Mint<mod> operator / (Mint<mod> o) { return *this * fexp(o, mod - 2); }

	friend ostream& operator << (ostream &os, const Mint<mod> &p) {
		return os << p.val;
	}

	friend istream& operator >> (istream &is, Mint<mod> &p) {
		return is >> p.val;
	}
};
\end{lstlisting}

\subsection{Linear Sieve of Eratosthenes
}
\begin{lstlisting}
// Can be used to calculate Multiplicative Functions, such as phi(n) (Mobius)

int phi[MAXN];
bool isPrime[MAXN];
vector<int> primes;

void sieve(int n = MAXN - 1) {
	for (int i = 1; i <= n; i++) {
		isPrime[i] = true;
		phi[i] = 1;
	}
	for (int i = 2; i <= n; i++) {
		if (isPrime[i]) {
			primes.push_back(i);
			phi[i] = -1;
		}
		for (auto p : primes) {
			if (1ll * p * i > n)
				break;
			isPrime[i * p] = false;
			if (i % p == 0) {
				phi[i * p] = 0;
				break;
			} else {
				phi[i * p] = -phi[i];
			}
		}
	}
}
\end{lstlisting}

\subsection{Matrix
}
\begin{lstlisting}
// It's preferable to declare global matrices
// based on: https://github.com/tfg50/Competitive-Programming/blob/master/Biblioteca/Math/Matrix.cpp
template <const int n, const int m, class T = Mint<>>
struct Matrix {
	T mat[n][m];

	Matrix(int d = 0) {
		for (int i = 0; i < n; i++) {
			for (int j = 0; j < m; j++) {
				v[i][j] = T(0);
			}
			if (i < m)
				v[i][i] = T(d);
		}
	}

	template <int p>
	Matrix<n, p, T> operator * (const Matrix<m, p, T> &o) {
		Matrix<n, p, T> ans;
		for (int i = 0; i < n; i++) {
			for (int j = 0; j < p; j++) {
				for (int k = 0; k < m; k++) {
					ans.mat[i][j] = ans.mat[i][j] + mat[i][k] * o.mat[k][j];
				}
			}
		}
		return ans;
	}
};
\end{lstlisting}

\subsection{Miller Rabin e Pollard Rho
}
\begin{lstlisting}
// copied from: https://github.com/tfg50/Competitive-Programming/blob/master/Biblioteca/Math/MillerRho.cpp

//miller_rabin
typedef unsigned long long ull;
typedef long double ld;

ull fmul(ull a, ull b, ull m) {
	ull q = (ld) a * (ld) b / (ld) m;
	ull r = a * b - q * m;
	return (r + m) % m;
}

ull fexp(ull x, ull e, ull m) {
	ull ans = 1;
	x = x % m;
	for(; e; e >>= 1) {
		if(e & 1) {
			ans = fmul(ans, x, m);
		}
		x = fmul(x, x, m);
	}
	return ans;
}

bool miller(ull p, ull a) {
	ull s = p - 1;
	while(s % 2 == 0) s >>= 1;
	while(a >= p) a >>= 1;
	ull mod = fexp(a, s, p);
	while(s != p - 1 && mod != 1 && mod != p - 1) {
		mod = fmul(mod, mod, p);
		s <<= 1;
	}
	if(mod != p - 1 && s % 2 == 0)return false;
	else return true;
}

bool prime(ull p) {
	if(p <= 3)
		return true;
	if(p % 2 == 0)
		return false;
	return miller(p, 2) && miller(p, 3)
		&& miller(p, 5) && miller(p, 7)
		&& miller(p, 11) && miller(p, 13)
		&& miller(p, 17) && miller(p, 19)
		&& miller(p, 23) && miller(p, 29)
		&& miller(p, 31) && miller(p, 37);
}

//pollard_rho

ull func(ull x, ull c, ull n) {
	return (fmul(x, x, n) + c) % n;
}

ull gcd(ull a, ull b) {
	if(!b) return a;
	else return gcd(b, a % b);
}

ull rho(ull n) {
	if(n % 2 == 0) return 2;
	if(prime(n)) return n;
	while(1) {
		ull c;
		do {
			c = rand() % n;
		} while(c == 0 || (c + 2) % n == 0);
		ull x = 2, y = 2, d = 1;
		ull pot = 1, lam = 1;
		do {
			if(pot == lam) {
				x = y;
				pot <<= 1;
				lam = 0;
			}
			y = func(y, c, n);
			lam++;
			d = gcd(x >= y ? x - y : y - x, n);
		} while(d == 1);
		if(d != n) return d;
	}
}

std::vector<ull> factors(ull n) {
	std::vector<ull> ans, rest, times;
	if(n == 1) return ans;
	rest.push_back(n);
	times.push_back(1);
	while(!rest.empty()) {
		ull x = rho(rest.back());
		if(x == rest.back()) {
			int freq = 0;
			for(int i = 0; i < rest.size(); i++) {
				int cur_freq = 0;
				while(rest[i] % x == 0) {
					rest[i] /= x;
					cur_freq++;
				}
				freq += cur_freq * times[i];
				if(rest[i] == 1) {
					std::swap(rest[i], rest.back());
					std::swap(times[i], times.back());
					rest.pop_back();
					times.pop_back();
					i--;
				}
			}
			while(freq--) {
				ans.push_back(x);
			}
			continue;
		}
		ull e = 0;
		while(rest.back() % x == 0) {
			rest.back() /= x;
			e++;
		}
		e *= times.back();
		if(rest.back() == 1) {
			rest.pop_back();
			times.pop_back();
		}
		rest.push_back(x);
		times.push_back(e);
	}
	return ans;
}
\end{lstlisting}



%%%%%%%%%%%%%%%%%%%%
%
% Strings
%
%%%%%%%%%%%%%%%%%%%%

\section{Strings}

\subsection{Aho Corasick
}
\begin{lstlisting}
// based on: https://github.com/tfg50/Competitive-Programming/blob/master/Biblioteca/Strings/Aho.cpp
template <const int ALPHA = 26, class T = string, const int OFFSET = 'a'>
struct Aho {
	struct Node {
		int nxt[ALPHA];
		int size;
		int link, elink;
		bool end;

		Node() {
			for (int i = 0; i < ALPHA; i++) {
				nxt[i] = 0;
			}
			size = 0;
			link = elink = 0;
			end = false;
			// initialize new stuff here
		}
		// add new stuff here
	};

	vector<Node> nodes;

	Aho() {
		nodes.push_back(Node());
	}

	template <class F>
	void goUp(int on, F f) {
		for (on = nodes[on].end ? on : nodes[on].elink; on > 0; on = nodes[on].elink) {
			f(nodes[on]);
		}
	}

	template <class C>
	int nextState(int on, C c) const {
		return nodes[on].nxt[c - OFFSET];
	}

	int add(const T &s) {
		int cur = 0;
		for (auto ch : s) {
			if (nodes[cur].nxt[ch - OFFSET] == 0) {
				nodes[cur].nxt[ch - OFFSET] = (int) nodes.size();
				nodes.push_back(Node());
				nodes.back().size = nodes[cur].size + 1;
			}
			cur = nodes[cur].nxt[ch - OFFSET];
		}
		// mark this node as the end of a word
		nodes[cur].end = true;
		return cur;
	}

	void build() {
		queue<int> q;
		q.push(0);
		while (!q.empty()) {
			int cur = q.front();
			q.pop();
			nodes[cur].elink = (nodes[nodes[cur].link].end ? nodes[cur].link : nodes[nodes[cur].link].elink);
			for (int i = 0; i < ALPHA; i++) {
				int &nxt = nodes[cur].nxt[i];
				if (nxt) {
					nodes[nxt].link = (cur == 0 ? 0 : nodes[nodes[cur].link].nxt[i]);
					q.push(nxt);
				} else {
					nxt = nodes[nodes[cur].link].nxt[i];
				}
			}
		}
	}
};
\end{lstlisting}

\subsection{Algoritmo Z
}
\begin{lstlisting}
// Classic problems using Zfunction:
// - String Matching Problem
// - Number of Different Substrings in O(n^2)
// - Find the root of a String
// - Search with at most 1 mistake in O(n)

template <class T>
struct ZFunc {
	// z[i] = lenght of the longest common preffix of v[0, n) and v[i, n)
	vector<int> z;
	ZFunc(const T &v): z((int) v.size()) {
		int n = (int) v.size(), l = 0, r = 0;
		if (!z.empty()) z[0] = n;
		for (int i = 1; i < n; i++) {
			if (i <= r) z[i] = min(z[i - l], r - i + 1);
			while (i + z[i] < n && v[i + z[i]] == v[z[i]]) z[i]++;
			if (r < i + z[i] - 1) l = i, r = i + z[i] - 1;
		}
	}
};
\end{lstlisting}

\subsection{Hash Polinomial
}
\begin{lstlisting}
struct Hash {
	const int p[2] = {337, 521}; // remember to change bases if needed
	const int m[2] = {(int) 1e9 + 7, (int) 1e9 + 9};

	int n;
	bool single;
	vector<int> h[2], pw[2];

	Hash(string &s, bool fs = false) {
		n = (int) s.size();
		single = fs;
		for (int k : {0, 1}) {
			h[k] = pw[k] = vector<int>(n);
			pw[k][0] = 1;
			h[k][0] = (int) s[0];
			for (int i = 1; i < n; i++) {
				h[k][i] = (1ll * h[k][i - 1]  * p[k] + 1ll * (int) s[i]) % m[k];
				pw[k][i] = 1ll * pw[k][i - 1] * p[k] % m[k];
			}
			if (single) break;
		}
	}

	pair<int, int> substring(int l, int r) {
		if (l > r)
			swap(l, r);
		pair<int, int> ans = {0, 0};
		for (int k : {0, 1}) {
			int val = h[k][r] - (l > 0 ? 1ll * h[k][l - 1] * pw[k][r - l + 1] % m[k] : 0);
			val %= m[k];
			if (val < 0)
				val += m[k];
			if (!k) ans.first = val;
			else ans.second = val;
			if (single) break;
		}
		return ans;
	}
};
\end{lstlisting}

\subsection{KMP - Knuth Morris Pratt
}
\begin{lstlisting}
template <class T>
vector<int> prefixFunction(const T &v) {
	vector<int> p((int) v.size(), 0);
	for (int i = 1; i < (int) v.size(); i++) {
		p[i] = p[i - 1];
		while (p[i] > 0 && v[p[i]] != v[i]) p[i] = p[p[i] - 1];
		if (v[p[i]] == v[i]) p[i]++;
	}
	return p;
}

// Assumes that there is a '#' at the end of pattern
template <class T, class F>
void match(const T &txt, const T &pat, const vector<int> &p, F f) {
	for (int i = 0, k = 0; i < (int) txt.size(); i++) {
		while (k > 0 && pat[k] != txt[i]) k = p[k - 1];
		if (pat[k] == txt[i]) k++;
		if (k + 1 == (int) pat.size()) {
			// a match was found!
			f(i - k + 1, i);
		}
	}
}
\end{lstlisting}

\subsection{Manacher
}
\begin{lstlisting}
// copied from: https://github.com/tfg50/Competitive-Programming/blob/master/Biblioteca/Strings/Manacher.cpp
// manacher[0][i + 1] is the length of matches of even length palindrome, starting from [i, i + 1]
// manacher[1][i] is the length of matches of odd length palindrome, starting from [i, i]
std::array<std::vector<int>, 2> manacher(const std::string& s) {
	int n = (int) s.size();
	std::array<std::vector<int>, 2> p = {std::vector<int>(n + 1), std::vector<int>(n)};
	for(int z = 0; z < 2; z++) for (int i = 0, l = 0, r = 0; i < n; i++) {
		int t = r - i + !z;
		if (i < r) p[z][i] = std::min(t, p[z][l + t]);
		int L = i - p[z][i], R = i + p[z][i] - !z;
		while (L >= 1 && R + 1 < n && s[L - 1] == s[R + 1])
			p[z][i]++, L--, R++;
		if (R > r) l = L, r = R;
	}
	return p;
}
\end{lstlisting}

\subsection{Suffix Array
}
\begin{lstlisting}
// by tfg50
// found at: https://github.com/tfg50/Competitive-Programming/blob/master/Biblioteca/Strings/SuffixArray.cpp
class SuffixArray {
public:
	template<class T>
	std::vector<int> buildSuffix(const T &array) {
		int n = array.size();
		std::vector<int> sa(n);
		for(int i = 0; i < n; i++) {
			sa[i] = i;
		}
		std::sort(sa.begin(), sa.end(), [&](int a, int b) { return array[a] < array[b]; });
		int cur = 0;
		std::vector<int> inv(n);
		std::vector<int> nxt(n);
		inv[sa[0]] = 0;
		for(int i = 1; i < n; i++) {
			inv[sa[i]] = (array[sa[i - 1]] != array[sa[i]] ? ++cur : cur);
		}
		cur++;
		for(int k = 0; cur < n; k++) {
			cur = 0;
			auxSort(sa, inv, 1 << k);
			for(int l = 0, r = 0; l < n; l = r, cur++) {
				while(r < n && getPair(inv, sa[l], 1 << k) == getPair(inv, sa[r], 1 << k)) {
					nxt[sa[r++]] = cur;
				}
			}
			nxt.swap(inv);
		}
		return sa;
	}

	template<class T>
	std::vector<int> buildLCP(const std::vector<int> &sa, const T &array) {
		int n = sa.size();
		std::vector<int> inv(n);
		for(int i = 0; i < n; i++) {
			inv[sa[i]] = i;
		}
		std::vector<int> lcp(n, 0);
		for(int i = 0, k = 0; i < n; i++) {
			if(inv[i] + 1 == n) {
				k = 0;
				continue;
			}
			int j = sa[inv[i] + 1];
			while(i + k < n && j + k < n && array[i + k] == array[j + k]) {
				k++;
			}
			lcp[inv[i]] = k;
			if(k > 0) {
				k--;
			}
		}
		return lcp;
	}
private:
	void auxSort(std::vector<int> &sa, const std::vector<int> &inv, int offset) {
		// O(nlogn) step, O(nlog^2n) total
		std::sort(sa.begin(), sa.end(), [&](int a, int b) { return getPair(inv, a, offset) < getPair(inv, b, offset); });
		// O(n) step, O(nlogn) total -- TO DO --
	}
	std::pair<int, int> getPair(const std::vector<int> &inv, int pos, int offset) {
		return std::pair<int, int>(inv[pos], pos + offset < (int) inv.size() ? inv[pos + offset] : -1);
	}
};
\end{lstlisting}

\pagebreak


%%%%%%%%%%%%%%%%%%%%
%
% Extra
%
%%%%%%%%%%%%%%%%%%%%

\section{Extra}

\subsection{hash.sh}
\begin{lstlisting}
# Para usar (hash das linhas [l1, l2]):
# ./hash.sh arquivo.cpp l1 l2
# vlw Bruno :)
sed -n $2','$3' p' $1 | sed '/^#w/d' | cpp -dD -P -fpreprocessed | tr -d '[:space:]' | md5sum | cut -c-6
\end{lstlisting}

\subsection{stress.sh}
\begin{lstlisting}
P=a
make ${P} ${P}2 gen || exit 1
for ((i = 1; ; i++)) do
	./gen $i > in
	./${P} < in > out
	./${P}2 < in > out2
	if (! cmp -s out out2) then
		echo "--> entrada:"
		cat in
		echo "--> saida1:"
		cat out
		echo "--> saida2:"
		cat out2
		break;
	fi
	echo $i
done
\end{lstlisting}

\subsection{template.cpp}
\begin{lstlisting}
#include <bits/stdc++.h>

using namespace std;
using ll = long long;
using pii = pair<int, int>;
using pll = pair<ll, ll>;

mt19937 rng((int) chrono::steady_clock::now().time_since_epoch().count());

const int MOD = 1e9 + 7;
const int MAXN = 2e5 + 5;
const ll INF = 1e18;

int main() {
	ios::sync_with_stdio(false);
	cin.tie(0);

	return 0;
}
\end{lstlisting}

\end{document}
